
% Default to the notebook output style

    


% Inherit from the specified cell style.




    
\documentclass[11pt]{article}

    
    
    \usepackage[T1]{fontenc}
    % Nicer default font (+ math font) than Computer Modern for most use cases
    \usepackage{mathpazo}

    % Basic figure setup, for now with no caption control since it's done
    % automatically by Pandoc (which extracts ![](path) syntax from Markdown).
    \usepackage{graphicx}
    % We will generate all images so they have a width \maxwidth. This means
    % that they will get their normal width if they fit onto the page, but
    % are scaled down if they would overflow the margins.
    \makeatletter
    \def\maxwidth{\ifdim\Gin@nat@width>\linewidth\linewidth
    \else\Gin@nat@width\fi}
    \makeatother
    \let\Oldincludegraphics\includegraphics
    % Set max figure width to be 80% of text width, for now hardcoded.
    \renewcommand{\includegraphics}[1]{\Oldincludegraphics[width=.8\maxwidth]{#1}}
    % Ensure that by default, figures have no caption (until we provide a
    % proper Figure object with a Caption API and a way to capture that
    % in the conversion process - todo).
    \usepackage{caption}
    \DeclareCaptionLabelFormat{nolabel}{}
    \captionsetup{labelformat=nolabel}

    \usepackage{adjustbox} % Used to constrain images to a maximum size 
    \usepackage{xcolor} % Allow colors to be defined
    \usepackage{enumerate} % Needed for markdown enumerations to work
    \usepackage{geometry} % Used to adjust the document margins
    \usepackage{amsmath} % Equations
    \usepackage{amssymb} % Equations
    \usepackage{textcomp} % defines textquotesingle
    % Hack from http://tex.stackexchange.com/a/47451/13684:
    \AtBeginDocument{%
        \def\PYZsq{\textquotesingle}% Upright quotes in Pygmentized code
    }
    \usepackage{upquote} % Upright quotes for verbatim code
    \usepackage{eurosym} % defines \euro
    \usepackage[mathletters]{ucs} % Extended unicode (utf-8) support
    \usepackage[utf8x]{inputenc} % Allow utf-8 characters in the tex document
    \usepackage{fancyvrb} % verbatim replacement that allows latex
    \usepackage{grffile} % extends the file name processing of package graphics 
                         % to support a larger range 
    % The hyperref package gives us a pdf with properly built
    % internal navigation ('pdf bookmarks' for the table of contents,
    % internal cross-reference links, web links for URLs, etc.)
    \usepackage{hyperref}
    \usepackage{longtable} % longtable support required by pandoc >1.10
    \usepackage{booktabs}  % table support for pandoc > 1.12.2
    \usepackage[inline]{enumitem} % IRkernel/repr support (it uses the enumerate* environment)
    \usepackage[normalem]{ulem} % ulem is needed to support strikethroughs (\sout)
                                % normalem makes italics be italics, not underlines
    

    
    
    % Colors for the hyperref package
    \definecolor{urlcolor}{rgb}{0,.145,.698}
    \definecolor{linkcolor}{rgb}{.71,0.21,0.01}
    \definecolor{citecolor}{rgb}{.12,.54,.11}

    % ANSI colors
    \definecolor{ansi-black}{HTML}{3E424D}
    \definecolor{ansi-black-intense}{HTML}{282C36}
    \definecolor{ansi-red}{HTML}{E75C58}
    \definecolor{ansi-red-intense}{HTML}{B22B31}
    \definecolor{ansi-green}{HTML}{00A250}
    \definecolor{ansi-green-intense}{HTML}{007427}
    \definecolor{ansi-yellow}{HTML}{DDB62B}
    \definecolor{ansi-yellow-intense}{HTML}{B27D12}
    \definecolor{ansi-blue}{HTML}{208FFB}
    \definecolor{ansi-blue-intense}{HTML}{0065CA}
    \definecolor{ansi-magenta}{HTML}{D160C4}
    \definecolor{ansi-magenta-intense}{HTML}{A03196}
    \definecolor{ansi-cyan}{HTML}{60C6C8}
    \definecolor{ansi-cyan-intense}{HTML}{258F8F}
    \definecolor{ansi-white}{HTML}{C5C1B4}
    \definecolor{ansi-white-intense}{HTML}{A1A6B2}

    % commands and environments needed by pandoc snippets
    % extracted from the output of `pandoc -s`
    \providecommand{\tightlist}{%
      \setlength{\itemsep}{0pt}\setlength{\parskip}{0pt}}
    \DefineVerbatimEnvironment{Highlighting}{Verbatim}{commandchars=\\\{\}}
    % Add ',fontsize=\small' for more characters per line
    \newenvironment{Shaded}{}{}
    \newcommand{\KeywordTok}[1]{\textcolor[rgb]{0.00,0.44,0.13}{\textbf{{#1}}}}
    \newcommand{\DataTypeTok}[1]{\textcolor[rgb]{0.56,0.13,0.00}{{#1}}}
    \newcommand{\DecValTok}[1]{\textcolor[rgb]{0.25,0.63,0.44}{{#1}}}
    \newcommand{\BaseNTok}[1]{\textcolor[rgb]{0.25,0.63,0.44}{{#1}}}
    \newcommand{\FloatTok}[1]{\textcolor[rgb]{0.25,0.63,0.44}{{#1}}}
    \newcommand{\CharTok}[1]{\textcolor[rgb]{0.25,0.44,0.63}{{#1}}}
    \newcommand{\StringTok}[1]{\textcolor[rgb]{0.25,0.44,0.63}{{#1}}}
    \newcommand{\CommentTok}[1]{\textcolor[rgb]{0.38,0.63,0.69}{\textit{{#1}}}}
    \newcommand{\OtherTok}[1]{\textcolor[rgb]{0.00,0.44,0.13}{{#1}}}
    \newcommand{\AlertTok}[1]{\textcolor[rgb]{1.00,0.00,0.00}{\textbf{{#1}}}}
    \newcommand{\FunctionTok}[1]{\textcolor[rgb]{0.02,0.16,0.49}{{#1}}}
    \newcommand{\RegionMarkerTok}[1]{{#1}}
    \newcommand{\ErrorTok}[1]{\textcolor[rgb]{1.00,0.00,0.00}{\textbf{{#1}}}}
    \newcommand{\NormalTok}[1]{{#1}}
    
    % Additional commands for more recent versions of Pandoc
    \newcommand{\ConstantTok}[1]{\textcolor[rgb]{0.53,0.00,0.00}{{#1}}}
    \newcommand{\SpecialCharTok}[1]{\textcolor[rgb]{0.25,0.44,0.63}{{#1}}}
    \newcommand{\VerbatimStringTok}[1]{\textcolor[rgb]{0.25,0.44,0.63}{{#1}}}
    \newcommand{\SpecialStringTok}[1]{\textcolor[rgb]{0.73,0.40,0.53}{{#1}}}
    \newcommand{\ImportTok}[1]{{#1}}
    \newcommand{\DocumentationTok}[1]{\textcolor[rgb]{0.73,0.13,0.13}{\textit{{#1}}}}
    \newcommand{\AnnotationTok}[1]{\textcolor[rgb]{0.38,0.63,0.69}{\textbf{\textit{{#1}}}}}
    \newcommand{\CommentVarTok}[1]{\textcolor[rgb]{0.38,0.63,0.69}{\textbf{\textit{{#1}}}}}
    \newcommand{\VariableTok}[1]{\textcolor[rgb]{0.10,0.09,0.49}{{#1}}}
    \newcommand{\ControlFlowTok}[1]{\textcolor[rgb]{0.00,0.44,0.13}{\textbf{{#1}}}}
    \newcommand{\OperatorTok}[1]{\textcolor[rgb]{0.40,0.40,0.40}{{#1}}}
    \newcommand{\BuiltInTok}[1]{{#1}}
    \newcommand{\ExtensionTok}[1]{{#1}}
    \newcommand{\PreprocessorTok}[1]{\textcolor[rgb]{0.74,0.48,0.00}{{#1}}}
    \newcommand{\AttributeTok}[1]{\textcolor[rgb]{0.49,0.56,0.16}{{#1}}}
    \newcommand{\InformationTok}[1]{\textcolor[rgb]{0.38,0.63,0.69}{\textbf{\textit{{#1}}}}}
    \newcommand{\WarningTok}[1]{\textcolor[rgb]{0.38,0.63,0.69}{\textbf{\textit{{#1}}}}}
    
    
    % Define a nice break command that doesn't care if a line doesn't already
    % exist.
    \def\br{\hspace*{\fill} \\* }
    % Math Jax compatability definitions
    \def\gt{>}
    \def\lt{<}
    % Document parameters
    \title{Demonstration}
    
    
    

    % Pygments definitions
    
\makeatletter
\def\PY@reset{\let\PY@it=\relax \let\PY@bf=\relax%
    \let\PY@ul=\relax \let\PY@tc=\relax%
    \let\PY@bc=\relax \let\PY@ff=\relax}
\def\PY@tok#1{\csname PY@tok@#1\endcsname}
\def\PY@toks#1+{\ifx\relax#1\empty\else%
    \PY@tok{#1}\expandafter\PY@toks\fi}
\def\PY@do#1{\PY@bc{\PY@tc{\PY@ul{%
    \PY@it{\PY@bf{\PY@ff{#1}}}}}}}
\def\PY#1#2{\PY@reset\PY@toks#1+\relax+\PY@do{#2}}

\expandafter\def\csname PY@tok@w\endcsname{\def\PY@tc##1{\textcolor[rgb]{0.73,0.73,0.73}{##1}}}
\expandafter\def\csname PY@tok@c\endcsname{\let\PY@it=\textit\def\PY@tc##1{\textcolor[rgb]{0.25,0.50,0.50}{##1}}}
\expandafter\def\csname PY@tok@cp\endcsname{\def\PY@tc##1{\textcolor[rgb]{0.74,0.48,0.00}{##1}}}
\expandafter\def\csname PY@tok@k\endcsname{\let\PY@bf=\textbf\def\PY@tc##1{\textcolor[rgb]{0.00,0.50,0.00}{##1}}}
\expandafter\def\csname PY@tok@kp\endcsname{\def\PY@tc##1{\textcolor[rgb]{0.00,0.50,0.00}{##1}}}
\expandafter\def\csname PY@tok@kt\endcsname{\def\PY@tc##1{\textcolor[rgb]{0.69,0.00,0.25}{##1}}}
\expandafter\def\csname PY@tok@o\endcsname{\def\PY@tc##1{\textcolor[rgb]{0.40,0.40,0.40}{##1}}}
\expandafter\def\csname PY@tok@ow\endcsname{\let\PY@bf=\textbf\def\PY@tc##1{\textcolor[rgb]{0.67,0.13,1.00}{##1}}}
\expandafter\def\csname PY@tok@nb\endcsname{\def\PY@tc##1{\textcolor[rgb]{0.00,0.50,0.00}{##1}}}
\expandafter\def\csname PY@tok@nf\endcsname{\def\PY@tc##1{\textcolor[rgb]{0.00,0.00,1.00}{##1}}}
\expandafter\def\csname PY@tok@nc\endcsname{\let\PY@bf=\textbf\def\PY@tc##1{\textcolor[rgb]{0.00,0.00,1.00}{##1}}}
\expandafter\def\csname PY@tok@nn\endcsname{\let\PY@bf=\textbf\def\PY@tc##1{\textcolor[rgb]{0.00,0.00,1.00}{##1}}}
\expandafter\def\csname PY@tok@ne\endcsname{\let\PY@bf=\textbf\def\PY@tc##1{\textcolor[rgb]{0.82,0.25,0.23}{##1}}}
\expandafter\def\csname PY@tok@nv\endcsname{\def\PY@tc##1{\textcolor[rgb]{0.10,0.09,0.49}{##1}}}
\expandafter\def\csname PY@tok@no\endcsname{\def\PY@tc##1{\textcolor[rgb]{0.53,0.00,0.00}{##1}}}
\expandafter\def\csname PY@tok@nl\endcsname{\def\PY@tc##1{\textcolor[rgb]{0.63,0.63,0.00}{##1}}}
\expandafter\def\csname PY@tok@ni\endcsname{\let\PY@bf=\textbf\def\PY@tc##1{\textcolor[rgb]{0.60,0.60,0.60}{##1}}}
\expandafter\def\csname PY@tok@na\endcsname{\def\PY@tc##1{\textcolor[rgb]{0.49,0.56,0.16}{##1}}}
\expandafter\def\csname PY@tok@nt\endcsname{\let\PY@bf=\textbf\def\PY@tc##1{\textcolor[rgb]{0.00,0.50,0.00}{##1}}}
\expandafter\def\csname PY@tok@nd\endcsname{\def\PY@tc##1{\textcolor[rgb]{0.67,0.13,1.00}{##1}}}
\expandafter\def\csname PY@tok@s\endcsname{\def\PY@tc##1{\textcolor[rgb]{0.73,0.13,0.13}{##1}}}
\expandafter\def\csname PY@tok@sd\endcsname{\let\PY@it=\textit\def\PY@tc##1{\textcolor[rgb]{0.73,0.13,0.13}{##1}}}
\expandafter\def\csname PY@tok@si\endcsname{\let\PY@bf=\textbf\def\PY@tc##1{\textcolor[rgb]{0.73,0.40,0.53}{##1}}}
\expandafter\def\csname PY@tok@se\endcsname{\let\PY@bf=\textbf\def\PY@tc##1{\textcolor[rgb]{0.73,0.40,0.13}{##1}}}
\expandafter\def\csname PY@tok@sr\endcsname{\def\PY@tc##1{\textcolor[rgb]{0.73,0.40,0.53}{##1}}}
\expandafter\def\csname PY@tok@ss\endcsname{\def\PY@tc##1{\textcolor[rgb]{0.10,0.09,0.49}{##1}}}
\expandafter\def\csname PY@tok@sx\endcsname{\def\PY@tc##1{\textcolor[rgb]{0.00,0.50,0.00}{##1}}}
\expandafter\def\csname PY@tok@m\endcsname{\def\PY@tc##1{\textcolor[rgb]{0.40,0.40,0.40}{##1}}}
\expandafter\def\csname PY@tok@gh\endcsname{\let\PY@bf=\textbf\def\PY@tc##1{\textcolor[rgb]{0.00,0.00,0.50}{##1}}}
\expandafter\def\csname PY@tok@gu\endcsname{\let\PY@bf=\textbf\def\PY@tc##1{\textcolor[rgb]{0.50,0.00,0.50}{##1}}}
\expandafter\def\csname PY@tok@gd\endcsname{\def\PY@tc##1{\textcolor[rgb]{0.63,0.00,0.00}{##1}}}
\expandafter\def\csname PY@tok@gi\endcsname{\def\PY@tc##1{\textcolor[rgb]{0.00,0.63,0.00}{##1}}}
\expandafter\def\csname PY@tok@gr\endcsname{\def\PY@tc##1{\textcolor[rgb]{1.00,0.00,0.00}{##1}}}
\expandafter\def\csname PY@tok@ge\endcsname{\let\PY@it=\textit}
\expandafter\def\csname PY@tok@gs\endcsname{\let\PY@bf=\textbf}
\expandafter\def\csname PY@tok@gp\endcsname{\let\PY@bf=\textbf\def\PY@tc##1{\textcolor[rgb]{0.00,0.00,0.50}{##1}}}
\expandafter\def\csname PY@tok@go\endcsname{\def\PY@tc##1{\textcolor[rgb]{0.53,0.53,0.53}{##1}}}
\expandafter\def\csname PY@tok@gt\endcsname{\def\PY@tc##1{\textcolor[rgb]{0.00,0.27,0.87}{##1}}}
\expandafter\def\csname PY@tok@err\endcsname{\def\PY@bc##1{\setlength{\fboxsep}{0pt}\fcolorbox[rgb]{1.00,0.00,0.00}{1,1,1}{\strut ##1}}}
\expandafter\def\csname PY@tok@kc\endcsname{\let\PY@bf=\textbf\def\PY@tc##1{\textcolor[rgb]{0.00,0.50,0.00}{##1}}}
\expandafter\def\csname PY@tok@kd\endcsname{\let\PY@bf=\textbf\def\PY@tc##1{\textcolor[rgb]{0.00,0.50,0.00}{##1}}}
\expandafter\def\csname PY@tok@kn\endcsname{\let\PY@bf=\textbf\def\PY@tc##1{\textcolor[rgb]{0.00,0.50,0.00}{##1}}}
\expandafter\def\csname PY@tok@kr\endcsname{\let\PY@bf=\textbf\def\PY@tc##1{\textcolor[rgb]{0.00,0.50,0.00}{##1}}}
\expandafter\def\csname PY@tok@bp\endcsname{\def\PY@tc##1{\textcolor[rgb]{0.00,0.50,0.00}{##1}}}
\expandafter\def\csname PY@tok@fm\endcsname{\def\PY@tc##1{\textcolor[rgb]{0.00,0.00,1.00}{##1}}}
\expandafter\def\csname PY@tok@vc\endcsname{\def\PY@tc##1{\textcolor[rgb]{0.10,0.09,0.49}{##1}}}
\expandafter\def\csname PY@tok@vg\endcsname{\def\PY@tc##1{\textcolor[rgb]{0.10,0.09,0.49}{##1}}}
\expandafter\def\csname PY@tok@vi\endcsname{\def\PY@tc##1{\textcolor[rgb]{0.10,0.09,0.49}{##1}}}
\expandafter\def\csname PY@tok@vm\endcsname{\def\PY@tc##1{\textcolor[rgb]{0.10,0.09,0.49}{##1}}}
\expandafter\def\csname PY@tok@sa\endcsname{\def\PY@tc##1{\textcolor[rgb]{0.73,0.13,0.13}{##1}}}
\expandafter\def\csname PY@tok@sb\endcsname{\def\PY@tc##1{\textcolor[rgb]{0.73,0.13,0.13}{##1}}}
\expandafter\def\csname PY@tok@sc\endcsname{\def\PY@tc##1{\textcolor[rgb]{0.73,0.13,0.13}{##1}}}
\expandafter\def\csname PY@tok@dl\endcsname{\def\PY@tc##1{\textcolor[rgb]{0.73,0.13,0.13}{##1}}}
\expandafter\def\csname PY@tok@s2\endcsname{\def\PY@tc##1{\textcolor[rgb]{0.73,0.13,0.13}{##1}}}
\expandafter\def\csname PY@tok@sh\endcsname{\def\PY@tc##1{\textcolor[rgb]{0.73,0.13,0.13}{##1}}}
\expandafter\def\csname PY@tok@s1\endcsname{\def\PY@tc##1{\textcolor[rgb]{0.73,0.13,0.13}{##1}}}
\expandafter\def\csname PY@tok@mb\endcsname{\def\PY@tc##1{\textcolor[rgb]{0.40,0.40,0.40}{##1}}}
\expandafter\def\csname PY@tok@mf\endcsname{\def\PY@tc##1{\textcolor[rgb]{0.40,0.40,0.40}{##1}}}
\expandafter\def\csname PY@tok@mh\endcsname{\def\PY@tc##1{\textcolor[rgb]{0.40,0.40,0.40}{##1}}}
\expandafter\def\csname PY@tok@mi\endcsname{\def\PY@tc##1{\textcolor[rgb]{0.40,0.40,0.40}{##1}}}
\expandafter\def\csname PY@tok@il\endcsname{\def\PY@tc##1{\textcolor[rgb]{0.40,0.40,0.40}{##1}}}
\expandafter\def\csname PY@tok@mo\endcsname{\def\PY@tc##1{\textcolor[rgb]{0.40,0.40,0.40}{##1}}}
\expandafter\def\csname PY@tok@ch\endcsname{\let\PY@it=\textit\def\PY@tc##1{\textcolor[rgb]{0.25,0.50,0.50}{##1}}}
\expandafter\def\csname PY@tok@cm\endcsname{\let\PY@it=\textit\def\PY@tc##1{\textcolor[rgb]{0.25,0.50,0.50}{##1}}}
\expandafter\def\csname PY@tok@cpf\endcsname{\let\PY@it=\textit\def\PY@tc##1{\textcolor[rgb]{0.25,0.50,0.50}{##1}}}
\expandafter\def\csname PY@tok@c1\endcsname{\let\PY@it=\textit\def\PY@tc##1{\textcolor[rgb]{0.25,0.50,0.50}{##1}}}
\expandafter\def\csname PY@tok@cs\endcsname{\let\PY@it=\textit\def\PY@tc##1{\textcolor[rgb]{0.25,0.50,0.50}{##1}}}

\def\PYZbs{\char`\\}
\def\PYZus{\char`\_}
\def\PYZob{\char`\{}
\def\PYZcb{\char`\}}
\def\PYZca{\char`\^}
\def\PYZam{\char`\&}
\def\PYZlt{\char`\<}
\def\PYZgt{\char`\>}
\def\PYZsh{\char`\#}
\def\PYZpc{\char`\%}
\def\PYZdl{\char`\$}
\def\PYZhy{\char`\-}
\def\PYZsq{\char`\'}
\def\PYZdq{\char`\"}
\def\PYZti{\char`\~}
% for compatibility with earlier versions
\def\PYZat{@}
\def\PYZlb{[}
\def\PYZrb{]}
\makeatother


    % Exact colors from NB
    \definecolor{incolor}{rgb}{0.0, 0.0, 0.5}
    \definecolor{outcolor}{rgb}{0.545, 0.0, 0.0}



    
    % Prevent overflowing lines due to hard-to-break entities
    \sloppy 
    % Setup hyperref package
    \hypersetup{
      breaklinks=true,  % so long urls are correctly broken across lines
      colorlinks=true,
      urlcolor=urlcolor,
      linkcolor=linkcolor,
      citecolor=citecolor,
      }
    % Slightly bigger margins than the latex defaults
    
    \geometry{verbose,tmargin=1in,bmargin=1in,lmargin=1in,rmargin=1in}
    
    

    \begin{document}
    
    
    \maketitle
    
    

    
    \hypertarget{dynamic-refugee-matching}{%
\section{Dynamic Refugee Matching}\label{dynamic-refugee-matching}}

This notebook demonstrates the fuctioning of the mechanism introduced by
Andersson, Ehlers and Martinello (2018). All the necessary
documentation, requirements and dependecies should be documented in the
package. If you have any comment, spot any bug or some documentation is
missing, please let us know.

This notebook replicates and illustrates the allocation example we
provide in the paper (Tables 1 and 2). The purpose of this notebook is
to convey the intuition behind our allocation mechanism, and performs .
Moreover, this notebook is meant to introduce you to the simulations we
perform in the paper, which are replicable with a second notebook
(\texttt{simulations.ipynb}).

    \begin{Verbatim}[commandchars=\\\{\}]
{\color{incolor}In [{\color{incolor}1}]:} \PY{k+kn}{import} \PY{n+nn}{numpy} \PY{k}{as} \PY{n+nn}{np}
        \PY{k+kn}{import} \PY{n+nn}{scipy} \PY{k}{as} \PY{n+nn}{sp}
        \PY{k+kn}{import} \PY{n+nn}{pandas} \PY{k}{as} \PY{n+nn}{pd} 
        \PY{n}{pd}\PY{o}{.}\PY{n}{options}\PY{o}{.}\PY{n}{mode}\PY{o}{.}\PY{n}{chained\PYZus{}assignment} \PY{o}{=} \PY{k+kc}{None}
        \PY{n}{pd}\PY{o}{.}\PY{n}{set\PYZus{}option}\PY{p}{(}\PY{l+s+s2}{\PYZdq{}}\PY{l+s+s2}{display.max\PYZus{}rows}\PY{l+s+s2}{\PYZdq{}}\PY{p}{,} \PY{l+m+mi}{120}\PY{p}{)}
        \PY{n}{pd}\PY{o}{.}\PY{n}{set\PYZus{}option}\PY{p}{(}\PY{l+s+s2}{\PYZdq{}}\PY{l+s+s2}{display.max\PYZus{}columns}\PY{l+s+s2}{\PYZdq{}}\PY{p}{,} \PY{l+m+mi}{120}\PY{p}{)}
        
        \PY{o}{\PYZpc{}}\PY{k}{load\PYZus{}ext} autoreload
        \PY{o}{\PYZpc{}}\PY{k}{autoreload} 2
\end{Verbatim}


    \hypertarget{problem-description}{%
\subsection{Problem Description}\label{problem-description}}

In this section we replicate the allocation example provided in the
paper. The problem we want to tackle is to allocate a flow of \(N\)
refugees to \(M\) localities. Refugees arrive sequentially, and need to
be processed (and assigned) as they arrive.

As shown by Bansak et al. (2018) and Trapp et al. (2018), individual
refugees can have different probabilities of integrating in different
localities. We coarsen these probabilities into a binary indicator. That
is, every refugee can be either \emph{acceptable} or
\emph{non-acceptable} for a given locality. We indicate acceptability in
the refugee flows by a N\(\times\)M matrix, where each row represents a
refugee and each column a locality. This matrix will be populated by a
binary variable (0/1) indicating acceptability. We call this matrix
\textbf{scoring matrix}, or \texttt{scores}.

Localities might be subject to capacity constraints, or might be of
different sizes and thus subject to a quotas system. The algorithm can
accomodate these constraints trough a 1\(\times\)M \textbf{quotas
vector}, or \texttt{quotas}. This vector is a series of integers,
denoting the maximum number of refugees that can be assigned to each
locality \(m\).

\hypertarget{sequential-assignment-benchmark}{%
\subsection{Sequential assignment
(benchmark)}\label{sequential-assignment-benchmark}}

Throughout the paper we benchmark the performance of our assignment
mechanism with that of a naive sequential assignment rule. Sequrntial
assignment rules typically outperforms truly random assignments as in
the absence of quotas they guarantee that the number of asylum seekers
assigned to each locality differs by at most one.

To familiarize with scoring and assignment matrixes, we begin by
providing an example of sequential assignment, with and without quotas.
We define the \textbf{scoring matrix} as the transpose of Table 1 of the
paper. This matrix lists asylum seekers \(i\in{1,\ldots,N}\) by row, and
localities by column. A locality-asylum seeker match \((i,j)\) is
considered acceptable iff element \((i,j)\) of the coring matrix is
equal to 1. So the first two seekers are unacceptable in all three
localities, seekers three and four are acceptable in all localities, and
seeker 8 is acceptable only in localities 2 and 3.

We can optionally define a quotas array denoting the maximum capacity of
each locality. Here, locality 2 is the largest and can accomodate at
most 6 seekers. The sum of capacities should be greater than or equal to
the number of asylum seekers in the flow \(N\).

    \begin{Verbatim}[commandchars=\\\{\}]
{\color{incolor}In [{\color{incolor}2}]:} \PY{k+kn}{from} \PY{n+nn}{dynamic\PYZus{}refugee\PYZus{}matching}\PY{n+nn}{.}\PY{n+nn}{assignment} \PY{k}{import} \PY{n}{assign\PYZus{}seq}\PY{p}{,} \PY{n}{assign\PYZus{}random}
        \PY{c+c1}{\PYZsh{} Input scoring matrix}
        \PY{n}{scores\PYZus{}example} \PY{o}{=} \PY{n}{np}\PY{o}{.}\PY{n}{array}\PY{p}{(}  \PY{p}{[}\PY{p}{[}\PY{l+m+mi}{0}\PY{p}{,} \PY{l+m+mi}{0}\PY{p}{,} \PY{l+m+mi}{0}\PY{p}{]}\PY{p}{,}
                                     \PY{p}{[}\PY{l+m+mi}{0}\PY{p}{,} \PY{l+m+mi}{0}\PY{p}{,} \PY{l+m+mi}{0}\PY{p}{]}\PY{p}{,}
                                     \PY{p}{[}\PY{l+m+mi}{1}\PY{p}{,} \PY{l+m+mi}{1}\PY{p}{,} \PY{l+m+mi}{1}\PY{p}{]}\PY{p}{,}
                                     \PY{p}{[}\PY{l+m+mi}{1}\PY{p}{,} \PY{l+m+mi}{1}\PY{p}{,} \PY{l+m+mi}{1}\PY{p}{]}\PY{p}{,}
                                     \PY{p}{[}\PY{l+m+mi}{0}\PY{p}{,} \PY{l+m+mi}{0}\PY{p}{,} \PY{l+m+mi}{0}\PY{p}{]}\PY{p}{,}
                                     \PY{p}{[}\PY{l+m+mi}{0}\PY{p}{,} \PY{l+m+mi}{0}\PY{p}{,} \PY{l+m+mi}{0}\PY{p}{]}\PY{p}{,}
                                     \PY{p}{[}\PY{l+m+mi}{1}\PY{p}{,} \PY{l+m+mi}{1}\PY{p}{,} \PY{l+m+mi}{0}\PY{p}{]}\PY{p}{,}
                                     \PY{p}{[}\PY{l+m+mi}{0}\PY{p}{,} \PY{l+m+mi}{1}\PY{p}{,} \PY{l+m+mi}{1}\PY{p}{]}\PY{p}{,}
                                     \PY{p}{[}\PY{l+m+mi}{0}\PY{p}{,} \PY{l+m+mi}{0}\PY{p}{,} \PY{l+m+mi}{0}\PY{p}{]}\PY{p}{,}
                                     \PY{p}{[}\PY{l+m+mi}{1}\PY{p}{,} \PY{l+m+mi}{1}\PY{p}{,} \PY{l+m+mi}{1}\PY{p}{]}\PY{p}{]}\PY{p}{)}
        
        \PY{n}{quotas} \PY{o}{=} \PY{n}{np}\PY{o}{.}\PY{n}{array}\PY{p}{(}\PY{p}{[}\PY{l+m+mi}{2}\PY{p}{,}\PY{l+m+mi}{6}\PY{p}{,}\PY{l+m+mi}{3}\PY{p}{]}\PY{p}{)}
        
        \PY{c+c1}{\PYZsh{}\PYZsh{} Assign asylum seekers sequentially \PYZsh{}\PYZsh{}}
        \PY{n}{seq} \PY{o}{=} \PY{n}{assign\PYZus{}seq}\PY{p}{(}\PY{n}{scores\PYZus{}example}\PY{p}{)}
        \PY{n}{seq\PYZus{}quotas} \PY{o}{=} \PY{n}{assign\PYZus{}seq}\PY{p}{(}\PY{n}{scores\PYZus{}example}\PY{p}{,} \PY{n}{vector\PYZus{}quotas}\PY{o}{=}\PY{n}{quotas}\PY{p}{)}
        
        \PY{c+c1}{\PYZsh{}\PYZsh{} Print assignments \PYZsh{}\PYZsh{}}
        \PY{n+nb}{print}\PY{p}{(}\PY{l+s+s2}{\PYZdq{}}\PY{l+s+s2}{Sequential assigment without quotas}\PY{l+s+se}{\PYZbs{}n}\PY{l+s+s2}{\PYZdq{}}\PY{p}{,} \PY{n}{seq}\PY{o}{.}\PY{n}{assignment}\PY{p}{)}
        \PY{n+nb}{print}\PY{p}{(}\PY{l+s+s2}{\PYZdq{}}\PY{l+s+s2}{Sequential assigment with quotas}\PY{l+s+se}{\PYZbs{}n}\PY{l+s+s2}{\PYZdq{}}\PY{p}{,} \PY{n}{seq\PYZus{}quotas}\PY{o}{.}\PY{n}{assignment}\PY{p}{)}
\end{Verbatim}


    \begin{Verbatim}[commandchars=\\\{\}]
Sequential assigment without quotas
 [[1 0 0]
 [0 1 0]
 [0 0 1]
 [1 0 0]
 [0 1 0]
 [0 0 1]
 [1 0 0]
 [0 1 0]
 [0 0 1]
 [1 0 0]]
Sequential assigment with quotas
 [[1 0 0]
 [0 1 0]
 [0 0 1]
 [1 0 0]
 [0 1 0]
 [0 0 1]
 [0 1 0]
 [0 0 1]
 [0 1 0]
 [0 1 0]]

    \end{Verbatim}

    The assigment matrix is of the same shape of the scoring matrix,
\(N\times M\). Each row sums to one, and cells (i,j) such that
\texttt{assignment{[}i,j{]}=0} denote assignment of asylum seeker \(i\)
to locality \(j\). The assignment rule here i easy to grasp. The
algorithm assigns asylum seekers sequentially until all capacities are
filled. In fact, this algorithm does not take the contents of the coring
matrix under consideration at all (we only need to input it to measure
\(N\) and \(M\)).

However, note that in this particular example the assignment chosen by
the naive sequential algorithm is efficient, as all acceptable asylum
seekers are assigned to a municipality that finds them acceptable. In
other words, the sum of all successfully matched asylum seekers is equal
to the number of demanded and over-demanded asylum seekers

    \begin{Verbatim}[commandchars=\\\{\}]
{\color{incolor}In [{\color{incolor}3}]:} \PY{k+kn}{from} \PY{n+nn}{dynamic\PYZus{}refugee\PYZus{}matching}\PY{n+nn}{.}\PY{n+nn}{evaluate} \PY{k}{import} \PY{n}{evaluate\PYZus{}efficiency\PYZus{}case}
        \PY{n}{evaluate\PYZus{}efficiency\PYZus{}case}\PY{p}{(}\PY{n}{scores\PYZus{}example}\PY{p}{,} \PY{n}{seq}\PY{p}{)}
\end{Verbatim}


    \begin{Verbatim}[commandchars=\\\{\}]
Sum of demanded and over-demanded asylum seekers : 5 
Sum of well-matched asylum seekers               : 5

    \end{Verbatim}

    \hypertarget{mechanism-with-rotation-aem}{%
\subsection{Mechanism with rotation
(AEM)}\label{mechanism-with-rotation-aem}}

We show here how our proposed mechanism would assign asylum seekers
characterized by matrix \texttt{scores\_example}. This part of the
notebook replicates Table 2 of the paper. Furthermore, we show how the
mechanism would assign refugee in the presence of (binding) capacity
constraints.

    \begin{Verbatim}[commandchars=\\\{\}]
{\color{incolor}In [{\color{incolor}4}]:} \PY{k+kn}{from} \PY{n+nn}{dynamic\PYZus{}refugee\PYZus{}matching}\PY{n+nn}{.}\PY{n+nn}{assignment} \PY{k}{import} \PY{n}{assign}
        \PY{c+c1}{\PYZsh{}\PYZsh{} Assign asylum seekers according to our rotation mechanism \PYZsh{}\PYZsh{}}
        \PY{n}{aem} \PY{o}{=} \PY{n}{assign}\PY{p}{(}\PY{n}{scores\PYZus{}example}\PY{p}{)}
        \PY{n}{aem\PYZus{}quotas} \PY{o}{=} \PY{n}{assign}\PY{p}{(}\PY{n}{scores\PYZus{}example}\PY{p}{,} \PY{n}{vector\PYZus{}quotas}\PY{o}{=}\PY{n}{quotas}\PY{p}{)}
        
        \PY{c+c1}{\PYZsh{}\PYZsh{} Print assignments \PYZsh{}\PYZsh{}}
        \PY{n+nb}{print}\PY{p}{(}\PY{l+s+s2}{\PYZdq{}}\PY{l+s+s2}{AEM assigment without quotas}\PY{l+s+se}{\PYZbs{}n}\PY{l+s+s2}{\PYZdq{}}\PY{p}{,} \PY{n}{aem}\PY{o}{.}\PY{n}{assignment}\PY{p}{)}
        \PY{n+nb}{print}\PY{p}{(}\PY{l+s+s2}{\PYZdq{}}\PY{l+s+s2}{AEM assigment with quotas}\PY{l+s+se}{\PYZbs{}n}\PY{l+s+s2}{\PYZdq{}}\PY{p}{,} \PY{n}{aem\PYZus{}quotas}\PY{o}{.}\PY{n}{assignment}\PY{p}{)}
        
        \PY{c+c1}{\PYZsh{}\PYZsh{} Evaluate assignment \PYZsh{}\PYZsh{}}
        \PY{n}{evaluate\PYZus{}efficiency\PYZus{}case}\PY{p}{(}\PY{n}{scores\PYZus{}example}\PY{p}{,} \PY{n}{aem}\PY{p}{)}
\end{Verbatim}


    \begin{Verbatim}[commandchars=\\\{\}]
AEM assigment without quotas
 [[1 0 0]
 [0 1 0]
 [1 0 0]
 [0 1 0]
 [0 0 1]
 [1 0 0]
 [1 0 0]
 [0 0 1]
 [0 1 0]
 [0 1 0]]
AEM assigment with quotas
 [[1 0 0]
 [0 1 0]
 [1 0 0]
 [0 1 0]
 [0 0 1]
 [0 1 0]
 [0 1 0]
 [0 0 1]
 [0 0 1]
 [0 1 0]]
Sum of demanded and over-demanded asylum seekers : 5 
Sum of well-matched asylum seekers               : 5

    \end{Verbatim}

    Again (this time by construction), the final allocation is efficient.
However, \texttt{aem} also guarantees fairness. In the paper we define
fairness by the concept of envy. Theorem 1 in the paper shows that our
mechanism satisfies envy bounded by a single asylum seeker. No envy
means that no aggregation entity (locality), \emph{while being in the
market} (localities can exit the market with quotas), would like to
exchange its assigned asylum seekers with those of \textbf{any other
locality in the market}. With indivisible bundles, envy-free matching in
general des not exist. However, our algorithm guarantees envy bounded by
one: Envy between localities would disappear if they could obtain just
one refugee from another bundle.

Note that as municipalities have heterogenous preferences envy is not
symmetric. That is, in principle two localities can simultaneously envy
each other, or not envy each other at all. Unlike standard approaches
such as naive sequential assigments, our algorithm harnesses these
heterogeneities to ensure fair matchings.

We characterize envy after \(k\) arrivals with a \(M\times M\)
\textbf{envy matrix} \(E_{k}\). Each pair \(i,j\) indicates by how many
asylum seekers locality \(i\) (row) envies locality \(j\) (column). The
diagonal is naturally equal to zero. As Theorem 1 proves, in the
assignment produced by our algorithm, no locality envies another by more
than 1 asylum seekers (in this specific example, no locality envies any
other by any number of asylum seekers).

    \begin{Verbatim}[commandchars=\\\{\}]
{\color{incolor}In [{\color{incolor}5}]:} \PY{k+kn}{from} \PY{n+nn}{dynamic\PYZus{}refugee\PYZus{}matching}\PY{n+nn}{.}\PY{n+nn}{evaluate} \PY{k}{import} \PY{n}{calc\PYZus{}envy}
        
        \PY{n+nb}{print}\PY{p}{(}\PY{l+s+s1}{\PYZsq{}}\PY{l+s+s1}{Envy matrix after AEM assignment}\PY{l+s+se}{\PYZbs{}n}\PY{l+s+s1}{\PYZsq{}}\PY{p}{,} \PY{n}{calc\PYZus{}envy}\PY{p}{(}\PY{n}{aem}\PY{o}{.}\PY{n}{assignment}\PY{p}{,} \PY{n}{scores\PYZus{}example}\PY{p}{)}\PY{p}{)}
\end{Verbatim}


    \begin{Verbatim}[commandchars=\\\{\}]
Envy matrix after AEM assignment
 [[ 0  0 -2]
 [ 0  0  0]
 [-2  0  0]]

    \end{Verbatim}

    However, the naive sequential assignment does not achieve a fair
allocation. Specifically, by chance, the second locality envies the
first by 3 asylum seekers. In other words, the second locality would
need three more matches not to envy the first. Clearly, maintaining
pareto-efficiency, AEM selects a superior allocation.

    \begin{Verbatim}[commandchars=\\\{\}]
{\color{incolor}In [{\color{incolor}6}]:} \PY{k+kn}{from} \PY{n+nn}{dynamic\PYZus{}refugee\PYZus{}matching}\PY{n+nn}{.}\PY{n+nn}{evaluate} \PY{k}{import} \PY{n}{characterize\PYZus{}assignments}
        
        \PY{n+nb}{print}\PY{p}{(}\PY{l+s+s1}{\PYZsq{}}\PY{l+s+s1}{Envy matrix after sequential assignment}\PY{l+s+se}{\PYZbs{}n}\PY{l+s+s1}{\PYZsq{}}\PY{p}{,} \PY{n}{calc\PYZus{}envy}\PY{p}{(}\PY{n}{seq}\PY{o}{.}\PY{n}{assignment}\PY{p}{,} \PY{n}{scores\PYZus{}example}\PY{p}{)}\PY{p}{)}
\end{Verbatim}


    \begin{Verbatim}[commandchars=\\\{\}]
Envy matrix after sequential assignment
 [[ 0 -5 -3]
 [ 3  0  0]
 [ 1  0  0]]

    \end{Verbatim}

    Things get for the naive sequential assignment with time, as the number
of asylum seekers grows. To show this worsening of performance, we
duplicate the original asylum seeker flow by ten times, and compare the
performance of our mechanism with those of a sequential assigment rule
and random allocation.

With ten times more refugees, both the sequential and random allocation
mechanisms are not pareto-efficient, as they result in fewer than the
potential 50 appropriate matches in the simulated asylum seeker flows.
Moreover, while no locality envies another after allocating 100 asylum
seekers with \texttt{aem}, two out of three localities envy another with
an alternative uninformed assignment.

    \begin{Verbatim}[commandchars=\\\{\}]
{\color{incolor}In [{\color{incolor}7}]:} \PY{c+c1}{\PYZsh{} Set seed numpy}
        \PY{c+c1}{\PYZsh{} Replicate original arrival flow x10}
        \PY{k}{for} \PY{n}{i} \PY{o+ow}{in} \PY{n+nb}{range}\PY{p}{(}\PY{l+m+mi}{10}\PY{p}{)}\PY{p}{:}
            \PY{k}{if} \PY{n}{i} \PY{o}{==} \PY{l+m+mi}{0}\PY{p}{:}
                \PY{n}{scores\PYZus{}large} \PY{o}{=} \PY{n}{scores\PYZus{}example}
            \PY{k}{else}\PY{p}{:}
                \PY{n}{scores\PYZus{}large} \PY{o}{=} \PY{n}{np}\PY{o}{.}\PY{n}{vstack}\PY{p}{(}\PY{p}{(}\PY{n}{scores\PYZus{}large}\PY{p}{,}\PY{n}{scores\PYZus{}example}\PY{p}{)}\PY{p}{)}
        
        \PY{c+c1}{\PYZsh{} Allocation quality summary}
        \PY{n}{assignments} \PY{o}{=} \PY{p}{\PYZob{}}
            \PY{l+s+s1}{\PYZsq{}}\PY{l+s+s1}{AEM}\PY{l+s+s1}{\PYZsq{}}\PY{p}{:}               \PY{n}{assign}\PY{p}{(}\PY{n}{scores\PYZus{}example}\PY{p}{)}\PY{p}{,}
            \PY{l+s+s1}{\PYZsq{}}\PY{l+s+s1}{Sequential}\PY{l+s+s1}{\PYZsq{}}\PY{p}{:}        \PY{n}{assign\PYZus{}seq}\PY{p}{(}\PY{n}{scores\PYZus{}example}\PY{p}{)}\PY{p}{,}
            \PY{l+s+s1}{\PYZsq{}}\PY{l+s+s1}{Random}\PY{l+s+s1}{\PYZsq{}}\PY{p}{:}            \PY{n}{assign\PYZus{}random}\PY{p}{(}\PY{n}{scores\PYZus{}example}\PY{p}{)}\PY{p}{,}
            \PY{l+s+s1}{\PYZsq{}}\PY{l+s+s1}{AEM (10x)}\PY{l+s+s1}{\PYZsq{}}\PY{p}{:}         \PY{n}{assign}\PY{p}{(}\PY{n}{scores\PYZus{}large}\PY{p}{)}\PY{p}{,}
            \PY{l+s+s1}{\PYZsq{}}\PY{l+s+s1}{Sequential (10x)}\PY{l+s+s1}{\PYZsq{}}\PY{p}{:}  \PY{n}{assign\PYZus{}seq}\PY{p}{(}\PY{n}{scores\PYZus{}large}\PY{p}{)}\PY{p}{,}
            \PY{l+s+s1}{\PYZsq{}}\PY{l+s+s1}{Random (10x)}\PY{l+s+s1}{\PYZsq{}}\PY{p}{:}      \PY{n}{assign\PYZus{}random}\PY{p}{(}\PY{n}{scores\PYZus{}large}\PY{p}{)}\PY{p}{,}
        \PY{p}{\PYZcb{}}
        \PY{n}{characterize\PYZus{}assignments}\PY{p}{(}\PY{n}{assignments}\PY{p}{)}
\end{Verbatim}


\begin{Verbatim}[commandchars=\\\{\}]
{\color{outcolor}Out[{\color{outcolor}7}]:}                                                   AEM  Sequential  Random  \textbackslash{}
        Sum of demanded and over-demanded asylum seekers    5           5       5   
        Sum of well-matched asylum seekers                  5           5       4   
        \# localities envying another by more than 1 AS      0           2       2   
        Maximum envy                                        0           3       3   
        
                                                          AEM (10x)  Sequential (10x)  \textbackslash{}
        Sum of demanded and over-demanded asylum seekers         50                50   
        Sum of well-matched asylum seekers                       50                44   
        \# localities envying another by more than 1 AS            0                 2   
        Maximum envy                                              0                 3   
        
                                                          Random (10x)  
        Sum of demanded and over-demanded asylum seekers            50  
        Sum of well-matched asylum seekers                          44  
        \# localities envying another by more than 1 AS               1  
        Maximum envy                                                 8  
\end{Verbatim}
            
    \hypertarget{references}{%
\subsection{References}\label{references}}

Andersson, T., L. Ehlers, and A. Martinello (2018). Dynamic Refugee
Matching. Lund University Department of Economics Working Paper 2018

Bansak, K., J. Ferwerda, J. Hainmueller, A. Dillon, D. Hangartner, and
D. Lawrence (2018). Improving refugee integration through data-driven
algorithmic assignment. Science 359, 325-329.

Trapp


    % Add a bibliography block to the postdoc
    
    
    
    \end{document}
