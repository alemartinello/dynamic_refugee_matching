\documentclass[12pt,fleqn]{article}
\usepackage{amsmath,amsfonts,amssymb,amsthm}
\usepackage{latexsym}
\usepackage{mdwlist}
\usepackage{a4wide}
\usepackage{ae,aecompl}
\usepackage{titlesec}
\usepackage{graphicx}
\usepackage{epstopdf}
\usepackage[singlelinecheck=false]{caption}

\newtheorem{theorem}{Theorem}
\newtheorem{definition}{Definition}
\newtheorem{lemma}{Lemma}
\newtheorem{proposition}{Proposition}
\newtheorem{corollary}{Corollary}
\newtheorem{example}{Example}
\newtheorem{remark}{Remark}

\newcommand{\Xomit}[1]{}
\renewcommand{\baselinestretch}{1.1}

\renewcommand{\rmdefault}{ptm}
\titleformat*{\section}{\large\bfseries}
\titleformat*{\subsection}{\normalsize\bfseries}

\begin{document}
\section{Introduction}
This document contains the basic model and definitions and some theoretical results related to fairness and efficiency. Note, in particular, that this is a fair allocation problem and for this reason, preferences are taken as given (the idea is to estimate the preferences, i.e., they not are self-reported by the municipalities or the asylum seekers) and non-manipulability is, consequently, not considered. Introduction and full description of the problem to be added later (as well as additional theoretical results). Note that the instructions from the Swedish Migration Agency are that the mechanism should be fair, efficient, computationally simple, easy to implement, and easy to understand. The mechanism presented in this document satisfies all these properties.

\section{The Model and Basic Definitions}\label{SEC:Model}
Asylum seekers arrive in a \emph{sequence} $S=(a_1,\ldots,a_S)$ within a predetermined \emph{time period}. One can think of this time period as ``the next year'' but all results presented in this paper are valid for a time period of arbitrary length (see the discussion after Definition \ref{DEF:Rotation}). The $k$th asylum seeker to arrive is denoted by $S(k)$ and is sometimes referred to as \emph{arrival} $k$. An asylum seeker is of a specific \emph{type}. The type specifies the characteristics of the asylum seeker, e.g., age, education, spoken languages, etc. The set of all possible types is $T=\{t_1,\ldots,t_T\}$. Throughout the paper, it is assumed that the number of asylum seekers that arrive within the time period (i.e., the cardinality of $S$) is known but that it is unknown exactly how many asylum seekers of each type that will arrive (i.e., the distribution of $T$).

The set of \emph{municipalities} is given by $M=\{1,\ldots,M\}$. Each municipality $m\in M$ have a \emph{quota} that describes the number of asylum seekers that will be assigned to the municipality within the time period. The quotas are gathered in the vector $Q=(q_1,\ldots, q_M)$ and are chosen in such a way that $\sum_{m\in Q}q_m=|S|$, i.e., such that all asylum seekers can be assigned to a municipality.

Not all municipalities and not all asylum seekers are relevant during the entire time period since asylum seekers arrive according to the sequence $S$ and because municipalities cannot be assigned additional asylum seekers once their quota is filled. To formalize this, let, for any given $k\in \{1,\ldots,S\}$, the set $M(k)$ contain all municipalities that not yet have filled their quota when asylum seeker $S(k)$ arrives, and let the set $A(k)$ contain all asylum seekers in the set $\{S(1),\ldots,S(k)\}$ that has been assigned to a municipality in $M(k)$. An \emph{economy} $\mathcal{E}(k)$ contains the municipalities in $M(k)$ and the asylum seekers in $A(k)$.

A \emph{bundle} $x_m(k)$ contains all asylum seekers in $A(k)$ that has been assigned to municipality $m$. An \emph{allocation} is a vector $x(k)$ containing the bundles of all municipalities in $M(k)$. An allocation is \emph{feasible} if no municipality have been assigned more asylum seekers that their quota.

Municipalities classify each type in $T$ as either \emph{acceptable} or \emph{unacceptable} and, consequently, also asylum seekers as either acceptable or unacceptable. This means that each municipality $m\in M(k)$ can partition all asylum seekers in any given bundle $x_i(k)$ into two disjoint sets $A^+_m(x_i(k))$ and $A^-_m(x_i(k))$ where the former set contains all acceptable asylum seekers in the bundle $x_i(k)$ and the latter set contains all unacceptable asylum seekers in the bundle $x_i(k)$.

Given the above type of partitioning, a municipality can also weakly rank any two bundles in a given economy $\mathcal{E}(k)$. This ranking is based on the number of acceptable and unacceptable asylum seekers in the two given bundles. More specifically, for any two bundles, $x_i(k)$ and $x_j(k)$, municipality $m$ strictly prefers bundle $x_i(k)$ to bundle $x_j(k)$ if and only if the difference between the number of acceptable and the number of unacceptable asylum seekers is larger in the former bundle than in the latter, i.e., if and only if:
\begin{equation*}
|A_m^+(x_i(k))|-|A_m^-(x_i(k))|>|A_m^+(x_j(k))|-|A_m^-(x_j(k))|.
\end{equation*}
\noindent Similarly, municipality $m$ is indifferent between the two bundles if and only if these differences coincide, i.e., if and only if:
\begin{equation*}
|A_m^+(x_i(k))|-|A_m^-(x_i(k))|=|A_m^+(x_j(k))|-|A_m^-(x_j(k))|.
\end{equation*}
\noindent If municipality $m\in M(k)$ weakly prefers bundle $x_i(k)$ to bundle $x_j(k)$, the notational convention $x_i(k)R_m(k) x_j(k)$ will be adopted to describe the relationship. The strict and indifference parts of $R_m(k)$ are denoted by $P_m(k)$ and $I_m(k)$, respectively.

It will sometimes be necessary to classify asylum seekers in terms of aggregated demand. For this purpose, an asylum seeker $S(k)$ that is defined to be \emph{non-demanded}, \emph{demanded}, and \emph{overdemanded}, if $S(k)$ is unacceptable for all municipalities in $M(k)$, acceptable for exactly one municipality in $M(k)$, and acceptable for strictly more that one municipality in $M(k)$, respectively.

Finally, a (deterministic dynamic) \emph{mechanism} is a rule $\varphi$ that for each $k\in\{1,\ldots,S\}$ assigns asylum seeker $S(k)$ to a municipality in $M(k)$ as soon as the asylum seeker arrives. A mechanism is feasible if it always selects a feasible allocation.

\section{Notions of Fairness and Efficiency}\label{SEC:Fair_Efficient}
Because the distribution of types in any given sequence of asylum seekers $S$ is \emph{ex ante} unknown, it is impossible to construct a mechanism that for any sequence $S$ always selects an allocation that satisfies a set of predetermined properties after the arrival of the last asylum seeker in the sequence. For this reason, the fairness and efficiency properties, considered in this paper, will be defined for a given economy $\mathcal{E}(k)$. This also means that predictions of future arrivals are not taken into consideration. There is a variety of plausible notions that can be used to evaluate a given allocation but the analysis in this paper will be restricted to two classical properties, namely envy-freeness and Pareto efficiency.

An allocation is \emph{envy-free} (Foley, 1967) if no municipality envies any other municipality at a given allocation. Modified for the considered dynamic setting, this property states that an allocation $x(k)$ is envy-free in a given economy $\mathcal{E}(k)$ if $x_{m}(k)R_m(k) x_{m^\prime}(k)$ for any $m,m^\prime\in M(k)$. It is well-known that envy-free allocations generally do not exist when objects are indivisible (as the asylum seekers in this problem) and in the absence of monetary transfers.\footnote{To see this, suppose that there are two municipalities and that the first asylum seeker that arrives (i.e., $S(1)$) is acceptable for both municipalities. In this case, the municipality that not is assigned the first asylum seeker will always envy the other municipality at allocation $x(1)$.} For this reason, the notion of envy-freeness will be slightly modified following Budish (2011). More specifically, in the remaining part of the paper, allocations that satisfy envy bounded by a single asylum seeker will be considered. This property means that whenever some municipality $m$ envies some municipality $m^\prime$, the envy can be ``eliminated'' by removing a single asylum seeker either from the bundle of municipality $m$ or from the bundle of municipality $m^\prime$.\footnote{Note that this definition is different from the corresponding definition in Budish (2011) since some asylum seekers may be unacceptable and, in this case, envy bounded by a single asylum seeker may be achieved by removing an unacceptable asylum seeker from the municipalities \emph{own} bundle. In Budish (2011), no agent is assigned unacceptable objects and it, consequently, suffices to remove acceptable objects from \emph{other} agents bundles.}
\begin{definition}\rm\label{DEF:1-Utility_DIFF}
For a given economy $\mathcal{E}(k)$, an allocation $x(k)$ satisfies envy bounded by a single asylum seeker if, for any $m,m^\prime\in M(k)$, at least one of the following conditions hold:
\begin{itemize}
\item[(i)] $x_m(k)R_m(k) x_{x^\prime}(k)$,
\item[(ii)] there exists some asylum seeker $a\in x_{m}(k)$ such that $x_m(k)\setminus\{a\}R_m(k) x_{m^\prime}(k)$,
\item[(iii)] there exists some asylum seeker $a^\prime\in x_{m^\prime}(k)$ such that $x_m(k)R_m(k) x_{m^\prime}(k)\setminus\{a^\prime\}$.
\end{itemize}
\end{definition}
\noindent The notion of envy bounded by a single asylum seeker is quite week in the sense that it doesn't reveal anything about the number of acceptable and unacceptable asylum seekers assigned to a specific municipality. To make this point clear, suppose that municipality $m$ experience that it has been assigned two acceptable and three unacceptable asylum seekers and that municipality $m$ experiences that some other municipality $m^\prime$ has been assigned five acceptable and five unacceptable asylum seekers. In this case, municipality $m$ experience that envy is bounded by a single asylum seeker. However, by isolating the asylum seekers and by only considering acceptable asylum seekers or by only considering unacceptable asylum seekers (again from the viewpoint of municipality $m$), it is clear that municipality $m$ envies municipality $m^\prime$ in terms of acceptable asylum seekers (if municipalities $m$ and $m^\prime$ share their views on unacceptable asylum seekers, then municipality $m^\prime$ envies municipality $m$ in terms of unacceptable asylum seekers).

To also evaluate envy in allocations from the perspective of only acceptable asylum seekers, the notion of envy bounded by a single acceptable asylum seeker will be adopted. This property is satisfied for municipality $m$ in relation to municipality $m^\prime$, if municipality $m$ experiences that it is assigned at most one fewer acceptable asylum seeker than municipality $m^\prime$ at a given allocation $x(k)$. The notion of envy bounded by a single unacceptable asylum seeker is defined in a corresponding fashion.

\begin{definition}\rm\label{DEF:1-Envy_ACC}
Consider an economy $\mathcal{E}(k)$, an allocation $x(k)$ and two municipalities $m,m^\prime\in M(k)$. Then municipality $m$ is unenvious of municipality $m^\prime$ in terms of acceptable asylum seekers if $|A_m^+(x_m(k))|\geq |A_m^+(x_{m^\prime}(k))|$, and envy towards municipality $m^\prime$ is bounded by a single acceptable asylum seeker if $|A_m^+(x_m(k))|\geq |A_m^+(x_{m^\prime}(k))|-1$.
\end{definition}

\begin{definition}\rm\label{DEF:1-Envy_UNACC}
Consider an economy $\mathcal{E}(k)$, an allocation $x(k)$ and two municipalities $m,m^\prime\in M(k)$. Then municipality $m$ is unenvious of municipality $m^\prime$ in terms of unacceptable asylum seekers if $|A_m^-(x_m(k))|\leq |A_m^-(x_{m^\prime}(k))|$, and envy towards $m^\prime$ is bounded by a single unacceptable asylum seeker if $|A_m^-(x_m(k))|-1\leq |A_m^-(x_{m^\prime}(k))|$.
\end{definition}
\noindent To evaluate efficiency properties of allocations, the notion of (ex post) Pareto efficiency is adopted. In the considered dynamic setting, this property states that an allocation $x(k)$ is Pareto efficient if there is no way to reallocate the asylum seekers that has been assigned to the municipalities in $M(k)$ among the municipalities in $M(k)$ in such a way that all quotas are respected, all municipalities experience that they are weakly better off and at least one municipality experience that it is strictly better off.
\begin{definition}\rm\label{DEF:Efficiency}
For a given economy $\mathcal{E}(k)$, an allocation $x(k)$ is (ex post) Pareto efficient if there is no way to reallocate the asylum seekers in $A(k)$ among the municipalities in $M(k)$ to obtain a new allocation $x^\prime(k)$ where the quotas are respected for all municipalities in $M(k)$, $x_m^\prime(k)R_m(k) x_m(k)$ for all $m\in M(k)$, and $x_m^\prime(k)P_m(k) x_m(k)$ for some $m\in M(k)$.
\end{definition}

\section{Structures and Properties of Structure Mechanisms}\label{SEC:Structure_Mechanisms}
Given that asylum seekers are assigned to municipalities directly upon arrival, conflicts between municipalities may arise because some asylum seekers are non-demanded and some are overdemanded. In both cases, there must be some rule that determines which municipality that the asylum seeker should be assigned to. These conflicts are resolved by means of priority and rejection structures where the former is used to resolve conflicts when an asylum seeker is overdemanded and the latter is adopted to assign non-demanded asylum seekers to municipalities.

A \emph{priority structure} is, for a given $k\in\{1,\ldots,S\}$, a list of strict orders $\pi(k)=\{\pi_t(k)\}_{t\in T}$ that determines the priorities among the municipalities over asylum seekers of different types in $T$. This means that if asylum seeker $S(k)$ is of type $t$, then the municipality with $\pi_t(k)=1$ has the highest priority over asylum seeker $S(k)$ among all municipalities in $M(k)$, the municipality with $\pi_t(k)=2$ has the second highest priority among all municipalities in $M(k)$, and so on.

A \emph{rejection structure} is, for a given $k\in\{1,\ldots,S\}$, a list of strict orders $\sigma(k)=\{\sigma_t(k)\}_{t\in T}$ where each \emph{rejection list} $\sigma_t(k)$ decides which municipality a non-demanded asylum seeker should be assigned to. More precisely, if asylum seeker $S(k)$ is of type $t$ and if all municipalities in $M(k)$ find asylum seeker $S(k)$ unacceptable, then $S(k)$ is assigned to the municipality with the highest position in the list $\sigma_t(k)$.

A \emph{structure} is a pair $(\pi(k),\sigma(k))$ where $\pi(k)$ is a priority-ordering and $\sigma(k)$ is a rejection list. Throughout the paper, it is assumed that the structure $(\pi(1),\sigma(1))$ is exogenously given, i.e., that the structures is in place when the first asylum seeker $S(1)$ arrives. A structure mechanism can be adopted to resolve conflicts of the above mentioned type.
\begin{definition}\rm\label{DEF:Structure_Mechanism}
A structure mechanism $\varphi$ is a rule that for a given structure $(\pi(k),\sigma(k))$ at a given $k\in\{1,\ldots,S\}$ selects an allocation $x(k)$ such that asylum seeker $S(k)$ is assigned to:
\begin{itemize}
\item[(i)] the municipality in $M(k)$ with the highest position on the list $\sigma_t(k)$ if $S(k)$ is non-demanded and of type $t$,
\item[(ii)] the only municipality in $M(k)$ that finds $S(k)$ acceptable if $S(k)$ is demanded,
\item[(iii)] the municipality in $M(k)$ with the highest priority in $\pi_t(k)$ that finds $S(k)$ acceptable if $S(k)$ is overdemanded and of type $t$.
\end{itemize}
\end{definition}

\noindent Uptil this point, structures has been generally defined and it has not explicitly been stated how the structure is ``updated'' between any two arrivals $S(k)$ and $S(k+1)$ (if updated at all). Given the interest in structure mechanisms and allocations where envy is bounded by a single asylum seeker, a first observation is that the priority \emph{and} the rejection structures must be updated between any two arrivals.
\begin{example}\rm
Let $M(1)=\{m_1,m_2\}$ and suppose that $(\pi(1),\sigma(1))$ is a structure with the property that municipality $m_1$ has the highest priority in $\pi_{t_1}(1)$ and $\pi_{t_2}(1)$ and municipality $m_2$ has the highest position in the rejection lists $\sigma_{t_1}(1)$ and $\sigma_{t_2}(1)$. Assume further that allocations are selected by a structure mechanism where \emph{only} the priority orderings are updated when an overdemanded asylum seeker is assigned to a municipality and \emph{only} the rejection lists are updated when a non-demanded asylum seeker is assigned to a municipality. Then if asylum seeker $S(1)$ is of type $t_1$ and acceptable for both municipalities, $S(1)$ must be assigned to municipality $m_1$, and if asylum seeker $S(2)$ is of type $t_2$ and non-demanded, $S(2)$ must be assigned to municipality $m_2$. But then envy is not bounded by a single asylum seeker at allocation $x(2)$ (the same conclusion holds if asylum seeker $S(1)$ is non-demanded and asylum seeker $S(2)$ is acceptable for both municipalities).\hfill$\square$
\end{example}

\noindent As illustrated in the above example, the priorities and the rejection lists cannot be treated separately from each other if one is interested in obtaining allocations where envy is bounded by a single asylum seeker. What is illustrated in the following example is that not even the various orderings in $\pi(k)$ (or in $\sigma(k)$ for that matter) can be treated separately from each other.

\begin{example}\rm
Let $M(1)=\{m_1,m_2\}$ and suppose that asylum seeker $S(k)$ is if type $t_k$ and that the first two arrivals are acceptable for both municipalities (i.e., rejection lists need not be considered). Suppose further that municipality $m_1$ has the highest priority in $\pi_{t_1}(1)$ and $\pi_{t_2}(1)$. Assume further that allocations are selected by a structure mechanism but that \emph{only} the priority structure for the specific type of arrival is updated between any two arrivals. In this case, both $S(1)$ and $S(2)$ must be assigned to municipality $m_1$. Consequently, envy is not bounded by a single asylum seeker at allocation $x(2)$.\hfill $\square$
\end{example}

\noindent The above two examples demonstrated that if municipalities not are sufficiently ``rewarded'' when assigned a non-demanded asylum seeker and not sufficiently ``penalized'' when assigned an overdemanded asylum seeker, then envy need not generally be bounded by a single asylum seeker. To find the right compromise between rewards and penalties when updating a structure between any two arrivals, the notion of rotation will be adopted.

To formalize the idea of rotation, let $\sigma^{O}(k)=(\sigma^O_t(k))_{t\in T}$ be a vector of positions where $\sigma^O_t(k)$ is the position of the municipality with the highest position in $\sigma_t(k)$ that envies municipality $m$ at allocation $x(k)$, and let $\sigma^{O}_t(k)=|M(k)|$ if no such municipality exists. Let further $\pi^{N}(k)=(\pi^N_t(k))_{t\in T}$ be a vector of positions where $\pi^N_t(k)$ is the position of the municipality with the highest position in $\pi_t(k)$ that is envied by municipality $m$ at allocation $x(k)$, and let $\pi^{N}_t(k)=|M(k)|$ if no such municipality exists.
\begin{definition}\rm\label{DEF:Rotation}
Suppose that asylum seeker $S(k)$ is of type $t\in T$ and assigned to municipality $m\in M(k)$. A structure $(\pi(k),\sigma(k))$ satisfies rotation if for any $k\in\{1,\ldots,S-1\}$:
\begin{enumerate}
\item[(i)] all municipalities in $M(k)\setminus \{m\}$ have the same priorities among themselves in $\pi_{t^\prime}(k+1)$ as in $\pi_{t^\prime}(k)$  for each $t^\prime\in T$,

\item[(ii)] all municipalities in $M(k)\setminus \{m\}$ have the same positions among themselves in $\sigma_{t^\prime}(k+1)$ as in $\sigma_{t^\prime}(k)$ for each $t^\prime\in T$,

\item[(iii)] if municipality $m$ not have filled the quota after after being assigned asylum seeker $S(k)$, then:
\begin{enumerate}
\item[(a)] if asylum seeker $S(k)$ is non-demanded, municipality $m$ is placed immediately before $\pi^{N}_t(k)$ in the priority ordering $\pi_t(k+1)$ for each $t\in T$, and gets the lowest position in the rejection list $\sigma_t(k+1)$ for each $t\in T$,
\item[(b)] if asylum seeker $S(k)$ is demanded, municipality $m$ have the same priority in the priority orderings $\pi_t(k+1)$ and $\pi_t(k)$ for each $t\in T$ as well as the same position in the rejection lists $\sigma_t(k+1)$ and $\sigma_t(k+1)$ for each $t\in T$,
\item[(c)]  if asylum seeker $S(k)$ is overdemanded, municipality $m$ is placed immediately before $\sigma^{O}_t(k)$ in the rejection list $\sigma_t(k+1)$ for each $t\in T$ and gets the lowest priority in $\pi_t(k+1)$ for each $t\in T$.\footnote{To see how $\pi^{N}_t(k)$ and $\sigma^{O}_t(k)$ should be interpreted, suppose that the list $\sigma_t(k)$ is given by $m_1<m_2<m_3<m_4$ for all $t\in T$, and that the ordering $\pi_t(k)$ is given by $m_2<m_3<m_1<m_4$ for all $t\in T$. If asylum seeker $S(k)$ is non-demanded and assigned to municipality $m_1$ and if $\pi^{N}_t(k)=2$, then $\pi_t(k+1)$ is given by $m_2<m_1<m_3<m_4$ for all $t\in T$ and $\sigma_t(k+1)$ is given by $m_2<m_3<m_4<m_1$ for all $t\in T$. Similarly, if asylum seeker $S(k)$ in overdemanded and assigned to municipality $m_1$ and if $\sigma^{O}_t(k)=4$, then $\pi_t(k+1)$ is given by $m_2<m_3<m_4<m_1$ for all $t\in T$ and $\sigma_t(k+1)$ is given by $m_2<m_3<m_1<m_4$ for all $t\in T$.}
\end{enumerate}
\end{enumerate}
\end{definition}
\noindent Before illustrating a structure mechanism that satisfies rotation (Example \ref{EX:Mechanism}), two remarks are in order. First, because all priority-orderings and all rejection lists have to be updated based on envy when an asylum seeker is non-demanded or overdemanded, it follows that all priority orderings and all rejection lists will be identical after sufficiently many arrivals. Second, municipalities with smaller quotas will fill their quotas before municipalities with larger quotas so it may well be the case that municipalities only are active during a small fraction of the sequence. If these issues are seen as a problem, they can be resolved by splitting the time period into smaller periods (i.e., instead of considering a one year period, one may divide the year in to quarters).
\begin{example}\rm\label{EX:Mechanism}
To illustrate the mechanics in a structure mechanism that satisfies rotation, suppose that $M(1)=\{m_1,m_2,m_3\}$ and $S=(1,\ldots,10,\ldots)$. Assume further that the quotas for all municipalities in $M(1)$ is greater than 10 and, in addition, that all priority-orderings and all rejection lists are identical at arrival $1$. The latter two assumptions means that one can abstract from the quotas in the first 10 arrivals as well as from type specific priority-orderings and rejection lists. Let now $\pi(1): m_1<m_2<m_3$ and $\sigma(1): m_1<m_2<m_3$, and suppose that the demand structures of the municipalities for the 10 first arrivals are given by Table \ref{TABLE:Demand} (the numbers 0 and 1 indicates that an asylum seeker is unacceptable and acceptable, respectively).
 
\begin{table}[h!]
\caption{Acceptable and unacceptable asylum seekers for the municipalities in Example \ref{TABLE:Demand}.}\label{TABLE:Demand}
\begin{tabular}{lllllllllll}\hline
$S$   & 1 & 2 & 3 & 4 & 5 & 6 & 7 & 8 & 9 & 10 \\ \hline
$m_1$ & 0 & 0 & 1 & 1 & 0 & 0 & 1 & 0 & 0 & 1\\
$m_2$ & 0 & 0 & 1 & 1 & 0 & 0 & 1 & 1 & 0 & 1\\
$m_3$ & 0 & 0 & 1 & 1 & 0 & 0 & 0 & 1 & 0 & 1\\ \hline
\end{tabular}
\end{table}

\noindent Because asylum seeker $S(1)$ is non-demanded and municipality $m_1$ has the highest position in the rejection list $\pi(1)$, asylum seeker $S(1)$ is assigned to municipality $m_1$. Consequently municipality $m_1$ envies both municipality $m_2$ and $m_3$ at allocation $x(1)$ and because municipality $m_2$ has a higher position than municipality $m_3$ in the priority-ordering $\pi(1)$ (i.e., position 2), it follows that $\pi^N(1)=2$. Hence, $\pi(2):m_1<m_2<m_3$. Furthermore, municipality $m_1$ is placed at the bottom of the rejection list $\sigma(2)$, i.e., $\sigma(2):m_2<m_3<m_1$. The entire process for the first 10 arrivals is described in Table \ref{TABLE:Mechanism}. From this table, it follows that, at allocation $x(10)$, asylum seekers $S(1)$, $S(3)$, $S(6)$ and $S(7)$ are assigned to municipality $m_1$, asylum seekers $S(2)$, $S(4)$, $S(9)$ and $S(10)$ are assigned to municipality $m_2$, and asylum seekers $S(5)$ and $S(8)$ are assigned to municipality $m_3$.\hfill $\square$

\begin{table}[h!]
\caption{Description of a structure mechanism that satisfies rotation for Example \ref{TABLE:Demand}.}\label{TABLE:Mechanism}
\begin{tabular}{llllll}\hline
$k$ & $S(k)$ assigned to $m_i$ & $\sigma^O(k)$ & $\pi^N(k)$ & $\sigma(k)$ & $\pi(k)$ \\ \hline
1   & $m_1$ & -- & 2 & $m_1<m_2<m_3$ & $m_1<m_2<m_3$ \\
2   & $m_2$ & -- & 3 & $m_2<m_3<m_1$ & $m_1<m_2<m_3$ \\
3   & $m_1$ & 3 & -- & $m_3<m_1<m_2$ & $m_1<m_2<m_3$ \\
4   & $m_2$ & 3 & -- & $m_3<m_1<m_2$ & $m_2<m_3<m_1$ \\
5   & $m_3$ & -- & 2 & $m_3<m_1<m_2$ & $m_3<m_1<m_2$ \\
6   & $m_1$ & -- & 3 & $m_1<m_2<m_3$ & $m_3<m_1<m_2$ \\
7   & $m_1$ & 3 & -- & $m_2<m_3<m_1$ & $m_3<m_1<m_2$ \\
8   & $m_3$ & 3 & -- & $m_2<m_3<m_1$ & $m_3<m_2<m_1$ \\
9   & $m_2$ & -- & 2 & $m_2<m_1<m_3$ & $m_2<m_1<m_3$ \\
10  & $m_2$ & 3 & -- & $m_1<m_3<m_2$ & $m_2<m_1<m_3$ \\ 
11  &       &   &    & $m_1<m_3<m_2$ & $m_1<m_3<m_2$ \\ \hline
\end{tabular}
\end{table}
\end{example}

\noindent The following result establishes that any structure mechanism where the structure satisfies rotation always selects a Pareto efficient allocation where envy is bounded by a single asylum seeker.

\begin{theorem}\rm\label{THEOREM:envy_efficiency}
Let $\varphi$ be a structure mechanism and $(\pi(k),\sigma(k))$ a structure that satisfies rotation for each $k\in \{1,\ldots,S\}$. If allocation $x(k)$ is selected by $\varphi$, then:
\begin{itemize}
\item[(i)] $x(k)$ satisfy envy bounded by a single asylum seeker,
\item[(ii)] $x(k)$ is Pareto efficient.
\end{itemize}
\end{theorem}
\begin{corollary}\rm\label{COROLLARY:envy}
Let $\varphi$ be a structure mechanism and $(\pi(k),\sigma(k))$ a structure that satisfies rotation for each $k\in \{1,\ldots,S\}$. If allocation $x(k)$ is selected by $\varphi$, then for any two municipalities $m,m^\prime\in M(k)$, it holds that municipality $m$ is unenvious of municipality $m^\prime$ in terms of acceptable asylum seekers and/or unenvious of municipality $m^\prime$ in terms of unacceptable asylum seekers.
\end{corollary}

\noindent Since a structure that satisfies rotation $(\pi(k),\sigma(k))$ is updated based on the selected allocation $x(k)$, it is possible to identify some very specific envy patterns in any structure $(\pi(k+1),\sigma(k+1))$ as reported in the following theorem.
\begin{theorem}\rm\label{TH:structures}
Let $\varphi$ be a structure mechanism and $(\pi(k),\sigma(k))$ a structure that satisfies rotation for each $k\in \{1,\ldots,S\}$. Then for each
$k\in\{1,\ldots,S\}$, each $m\in M(k)$ and each $t\in T$, it holds that municipality $m$ does not envy any
municipality with a higher priority in $\pi_t(k+1)$ or any municipality with lower position in $\sigma_t(k+1)$.
\end{theorem}

\section{Some Remarks}

\begin{itemize}
\item Note that it is assumed that each asylum seeker is of size one, i.e., that no families arrive. This needs to be modified but it does not matter for the description of the problem (we can just introduce some flexibility in the quotas).
\item Note that this is a fair allocation problem. In these problems, strategy-proofness is typically not considered.
\end{itemize}

\section*{Appendix: Proofs}
This Appendix contains the proofs of all results in the paper.

\medskip

\noindent\textbf{Proof of Theorem \ref{THEOREM:envy_efficiency}.} To prove Part (i) of the theorem, suppose that envy not is bounded by a single asylum seeker at $x(k)$. This means that there are two municipalities in $M(k)$, say $m$ and $m^\prime$, and two integers, say $k^0$ and $k^1$ (where $0<k^0<k^1\leq k$), such that:
\begin{itemize}
\item[(I)] municipality $m$ is indifferent between bundles $x_{m}(k^0-1)$ and $x_{m^\prime}(k^0-1)$,
\item[(II)] municipality $m$ envies municipality $m^\prime$ by exactly one asylum seeker at allocation $x(k^0)$, and
\item[(III)] municipality $m$ envies municipality $m^\prime$ by exactly two asylum seekers at allocation $x(k^1)$.
\end{itemize}
\noindent Let $k^0$ and $k^1$ be the largest and the smallest integers, respectively, that satisfies (I)--(III), and consider the subsequence of $S$ containing the asylum seekers $S(k^0),\ldots,S(k^1)$. Because the integers $k^0$ and $k^1$ are chosen in the above way, municipality $m$ has not been assigned any asylum seekers in the subsequence $S(k^0+1),\ldots,S(k^1-1)$, and municipality $m^\prime$ has not been assigned any non-demanded asylum seeker or any asylum seeker that is acceptable for municipality $m$ in the subsequence $S(k^0+1),\ldots,S(k^1-1)$. Hence, there may be four different reasons to why conditions (I)--(III) hold:
\begin{itemize}
\item[(a)] Asylum seekers $S(k^0)$ and $S(k^1)$ are acceptable to municipality $m$ but are both assigned to municipality $m^\prime$,
\item[(b)] Asylum seeker $S(k^0)$ is non-demanded but assigned to municipality $m$ and asylum seeker $S(k^1)$ is acceptable to municipality $m$ but assigned to municipality $m^\prime$,
\item[(c)] Asylum seeker $S(k^0)$ is acceptable to municipality $m$ but assigned to municipality $m^\prime$ and asylum seeker $S(k^1)$ is non-demanded but assigned to municipality $m$,
\item[(d)] Asylum seekers $S(k^0)$ and $S(k^1)$ are non-demanded but are both assigned to municipality $m$.
\end{itemize}
\noindent It is only proved that parts (a) and (b) cannot hold as the corresponding proofs of parts (c) and (d) are almost identical. In the remaining part of this proof, it is assumed that asylum seeker $S(k^0)$ is of type $t^\prime$ and that asylum seeker $S(k^1)$ is of type $t^{\prime\prime}$.

To prove that (a) cannot hold, note that asylum seeker $S(k^0)$ is acceptable for both municipality $m$ and $m^\prime$. Because of this and rotation, it also follows that municipality $m^\prime$ has a higher priority than municipality $m$ in $\pi_{t^\prime}(k^0)$ and the lowest priority in $\pi_t(k^0+1)$ for each $t\in T$. Because asylum seeker $S(k^1)$ is assigned to municipality $m^\prime$, it must also be the case that municipality $m^\prime$ has a higher priority than municipality $m$ in $\pi_{t^{\prime\prime}}(k^1)$. The latter can, however, not occur. To see this, recall first that municipality $m$ not is assigned any asylum seeker in the subsequence $S(k^0),\ldots,S(k^1)$. Hence, it cannot be the case that $m^\prime$ has a higher priority than municipality $m$ in $\pi_{t^{\prime\prime}}(k^1)$ because municipality $m$ is assigned an asylum seeker. The reason must then be that municipality $m^\prime$ is assigned a non-demanded asylum seeker since this is the only way for municipality $m^\prime$ to obtain a higher priority than $m$. But this contradicts the above conclusion that municipality $m^\prime$ has not been assigned any non-demanded asylum seeker in the subsequence $S(k^0+1),\ldots,S(k^1-1)$. Hence, (a) cannot hold.

To prove that condition (b) cannot hold, recall that asylum seeker $S(k^0)$ is non-demanded and that municipality $m$ envies municipality $m^\prime$ at allocation $x(k^0)$. Because of this and rotation, it also follows that municipality $m$ has a higher priority than municipality $m^\prime$ in $\pi_t(k^0+1)$ for each $t\in T$. Because the overdemanded asylum seeker $S(k^1)$ is assigned to municipality $m^\prime$, it must be the case that municipality $m^\prime$ has a higher priority than municipality $m$ in $\pi_{t^{\prime\prime}}(k^1)$. The latter can, however, not occur due to the rules of the mechanism. To see this, recall first that municipality $m$ not is assigned any asylum seeker in the subsequence $S(k^0),\ldots,S(k^1)$. Hence, it cannot be the case that $m^\prime$ has a higher priority than municipality $m$ in $\pi_{t^{\prime\prime}}(k^1)$ because municipality is assigned an asylum seeker. The reason must then be that municipality $m^\prime$ is assigned a non-demanded asylum seeker since this is the only way for municipality $m^\prime$ to obtain a higher priority than $m$. But this contradicts the above conclusion that municipality $m^\prime$ has not been assigned any non-demanded asylum seeker in the subsequence $S(k^0+1),\ldots,S(k^1-1)$.

To prove Part (ii) of the theorem, suppose that $x(k)$ not is Pareto efficient. Consider next the municipalities in the set $M(k)$ and the asylum seekers in the set $A(k)$. Because $x(k)$ not is Pareto efficient, by assumption, there exists an allocation $x^\prime(k)$ where the quotas are respected for all municipalities in $M(k)$, $x_m^\prime(k)R_m(k) x_m(k)$ for all $m\in M(k)$, and $x_m^\prime(k)P_m(k) x_m(k)$ for some $m\in M(k)$. The latter conditions imply that:
\begin{equation}
\sum_{j\in M(k)}\left(|A_j^+(x_j^\prime)|-|A_j^-(x_j^\prime)|\right)>\sum_{j\in M(k)}\left(|A_j^+({x_j})|-|A_j^-({x_j})|\right).\label{EQ:not_efficient_A}
\end{equation}
\noindent Note next that a municipality in $M(k)$ can only by assigned an unacceptable asylum seeker if the asylum seeker also is unacceptable for all other municipalities in $M(k)$. This follows since $\varphi$ is a structure mechanism. But this also means that $\sum_{j\in M(k)}|A_j^-(x_j^\prime)|=\sum_{j\in M(k)}|A_j^-({x_j})|$. Consequently, condition (\ref{EQ:not_efficient_A}) reduces to:
\begin{equation}
\sum_{j\in M(k)}|A_j^+(x_j^\prime)|>\sum_{j\in M(k)}|A_j^+({x_j})|,\label{EQ:not_efficient_B}
\end{equation}
\noindent i.e., that the total number of acceptable asylum seekers assigned to the municipalities in $M(k)$ is greater at allocation $x^\prime(k)$ than at allocation $x(k)$. But this is not possible since all asylum seekers in $A(k)$ are assigned to some municipality in $M(k)$ which finds them acceptable by construction of the structure mechanism $\varphi$. Hence, $x(k)$ must be Pareto efficient. \hfill $\square$

\medskip

\noindent\textbf{Proof of Theorem \ref{TH:structures}.} Only the first part of the statement is proved since the second part of the theorem is based on symmetrical arguments. Asylum seeker $S(k)$ is furthermore assumed to be of type $t\in T$ in the remaining part of the proof. For convenience, it is also, without loss of generality, written $\pi_t(k)$ in the remaining part of this proof instead of the formally correct statement ``$\pi_t(k)$ for each $t\in T$''.

The result is proved by induction. It is first demonstrated that the result is true for $\pi_t(2)$. Three cases must be considered:

\begin{itemize}

\item[(1.a)] Asylum seeker $S(1)$ is non-demanded. Because $S(1)$ not is demanded by any municipality in $M(1)$, no
municipality in $M(1)$ will envy the municipality that is assigned $S(1)$ and the municipality that is
assigned $S(1)$ will envy all municipalities in $M(1)$. Because the structure $(\pi(1),\sigma(1))$ satisfies
rotation, the municipality that is assigned $S(1)$ will have the highest priority in $\pi_t(2)$. These arguments
show that no municipality in $M(1)$ envies any municipality with a higher priority in $\pi_t(2)$.

\item[(1.b)] Asylum seeker $S(1)$ is demanded. In this case, $S(1)$ is assigned to the only municipality that demands
$S(1)$ and no municipality will envy this municipality. Similarly, the municipality that is assigned $S(1)$ will also not envy any other municipality since $S(1)$ is acceptable. Because the structure $(\pi(1),\sigma(1))$ satisfies rotation,
it follows that $\pi_t(1)=\pi_t(2)$, and, consequently, no municipality in $M(1)$ envies any
municipality with a higher priority in $\pi_t(2)$.

\item[(1.c)] Asylum seeker $S(1)$ is overdemanded. $S(1)$ is assigned to the municipality in $M(1)$ with the highest
priority in $\pi_t(1)$ that finds asylum seeker $S(1)$ acceptable. This municipality will be envied by all
municipalities that find $S(1)$ acceptable and the municipality that is assigned $S(1)$ will not envy any
municipality in $M(1)$. Because the structure $(\pi(1),\sigma(1))$ satisfies rotation, the
municipality that is assigned $S(1)$ will have the lowest priority in $\pi_t(2)$. These arguments show that no
municipality in $M(1)$ envies any municipality with a higher priority in $\pi_t(2)$.

\end{itemize}

\noindent From the above arguments, we conclude that the statement is true for $\pi_t(2)$. Consider now the
induction assumption that the result holds for $\pi_t(k)$ for an arbitrary $k\in \{2,\ldots,S-1\}$. Given this
assumption, it remains to show that the result is true for $\pi_t(k)$. Again, three cases must be considered:

\begin{itemize}

\item[(k.a)] Asylum seeker $S(k)$ is non-demanded. Because $S(k)$ not is demanded by any municipality in $M(k)$
and because envy is bounded by a single asylum seeker at allocation $x(k-1)$ by Theorem \ref{THEOREM:envy_efficiency}, no
municipality in $M(k)$ will envy the municipality that is assigned $S(k)$ at allocation $x(k)$. Because the structure
$(\pi(k),\sigma(k))$ satisfies rotation, the municipality that is assigned $S(k)$ will have a higher priority
in $\pi_t(k+1)$ than all municipalities that the municipality envies at allocation $x(k)$. These arguments show that
no municipality in $M(k)$ envies any municipality with a higher priority in $\pi_t(k+1)$.

\item[(k.b)] Asylum seeker $S(k)$ is demanded. In this case, $S(k)$ is assigned to the only municipality that demands
$S(k)$. Because the structure $(\pi(k),\sigma(k))$ satisfies rotation, it must be the case that $\pi_t(k)=\pi_t(k+1)$. Consequently, the result holds for $\pi_t(k+1)$ since the result is true for $\pi_t(k)$ by the induction assumption.

\item[(k.c)] Asylum seeker $S(k)$ is overdemanded. $S(k)$ is assigned to the municipality in $M(k)$ with the highest
priority in $\pi(k)$ that finds asylum seeker $S(k)$ acceptable. Because envy is bounded by a single asylum seeker
at allocation $x(k-1)$ by Theorem \ref{THEOREM:envy_efficiency}, the municipality that is assigned the acceptable asylum seeker $S(k)$ will
not envy any municipality at allocation $x(k)$. Then because the structure $(\pi(k),\sigma(k))$ satisfies rotation,
the municipality that is assigned $S(k)$ will have the lowest priority in $\pi_t(k+1)$. But then result holds by the above arguments.

\end{itemize}

\noindent The above three cases, together with the result for $\pi_t(2)$ and the induction assumption, proves that
municipality $m$ does not envy any municipality with a higher priority in $\pi_t(k)$. \hfill $\square$

\medskip

\noindent\textbf{Proof of Corollary \ref{COROLLARY:envy}.} Because allocation $x(k)$ is selected by $\varphi$, it follows from Theorem \ref{THEOREM:envy_efficiency} that envy for municipality $m$ towards municipality $m^\prime$ is bounded by a single asylum seeker, i.e., that:
\begin{equation}
A_m^+(x_m(k))-A_m^-(x_m(k))+1\geq A_m^+(x_{m^\prime}(k))-A_m^-(x_{m^\prime}(k)).\label{EQ:COR_1}
\end{equation}
\noindent To obtain a contradiction, suppose now that the statement not is true. Then:
\begin{eqnarray}
&& A_m^+(x_{m^\prime}(k))>A_m^+(x_m(k)),\label{EQ:COR_2} \\
&& A_m^-(x_m(k))>A_m^-(x_{m^\prime}(k)).\label{EQ:COR_3}
\end{eqnarray}
\noindent Inequality (\ref{EQ:COR_2}) implies that $A_m^+(x_{m^\prime}(k))-1\geq A_m^+(x_m(k))$. This condition together with inequality (\ref{EQ:COR_1}) implies that $A_m^-(x_{m^\prime}(k))\geq A_m^-(x_m(k))$. But this inequality contradicts inequality (\ref{EQ:COR_3}).\hfill $\square$





\end{document}


\section*{References}

\begin{itemize}

\item[] Budish (2011). JPE

\item[] Foley (1967)

\end{itemize}
